% please run using
% lualatex --shell-escape ...tex

\documentclass{article}
\usepackage{latexsym, a4wide}
\usepackage[utf8]{inputenc}
\usepackage[T1]{fontenc}
\setlength{\topmargin}{-2.5cm}
\setlength{\oddsidemargin}{-1cm}
\setlength{\textwidth}{18cm}
\setlength{\textheight}{23.5cm}
\parindent0em
\parskip1ex

\usepackage{color}
\usepackage{amsfonts,amssymb,mathtools,amsthm,mathrsfs,bbm}
\newtheorem{theorem}{Theorem}
\newtheorem{proposition}{Proposition}[section]
\newtheorem{lemma}[proposition]{Lemma}
\newtheorem{corollary}[proposition]{Corollary}
\newtheorem{definition}[proposition]{Definition}
\theoremstyle{definition}
\newtheorem{remark}[proposition]{Remark}
\newtheorem{example}[proposition]{Example}
\newtheoremstyle{step}{3pt}{0pt}{}{}{\bf}{}{.5em}{}
\theoremstyle{step} \newtheorem{step}{Step}

\DeclareMathAlphabet{\mathpzc}{OT1}{pzc}{m}{it}
\newcommand{\smallu}{\mathpzc{u}}

\usepackage{minted}
\setminted{encoding=utf-8}
\usepackage{fontspec}
%\setmainfont{dejavu-serif}
\setmonofont[Scale=0.85]{dejavu-sans}
%\setmainfont{FreeSerif}
%\setmonofont{FreeMono}
\usepackage{xcolor, graphics}
\definecolor{mygray}{rgb}{0.97,0.97,0.97}
\definecolor{LightCyan}{rgb}{0.8784,1,1}
\newcommand{\leanline}[1]{\mintinline[breaklines, breakafter=_., breaklines]{Lean}{#1}}%\textsuperscript{\textcolor{gray}{\tiny(m)}}}
\newcommand{\leanlinecolor}{\mintinline[breaklines, breakafter=_., breaklines, bgcolor=LightCyan]{Lean}}
\usepackage{refcheck}

\usepackage[notcite,notref]{showkeys}
\usepackage{hyperref}
\usepackage{paracol}

\begin{document}
\title{\LARGE Separating classes of functions}

\maketitle Our goal is to show that any algebra of functions (defined
on a Polish space) which separates points is separating.

\begin{remark}[Notation]
  We will write $(E,r)$ for some extended pseudo-metric space, $\mathcal
    P(E)$ for the set of probability measures on the Borel
  $\sigma$-algebra on $E$, $\mathbbm k \in \{\mathbb R, \mathbb C\}$,
  and $\mathcal C_{b}(E, \mathbbm k)$ the set of $\mathbbm k$-valued
  bounded continuous functions on $E$. For some $\mathbf P \in \mathcal
    P(E)$ and $f \in \mathcal C_{b}(E, \mathbbm k)$, we let $\mathbf P[f]
    := \int f(x) \mathbf P(dx) \in \mathbbm k$ be the expectation.
\end{remark}

\section{Bounded pointwise convergence}

The following is a simple consequence of dominated convergence, and
often needed in probability theory.

\begin{paracol}{2}
  \phantom{1}
  \begin{definition}
    Let $E$ be some set and $f, f_1, f_2,...: E \to \mathbbm k$. We say
    that $f_1,f_2,..:$ converges to $f$ boundedly pointwise if $f_n
      \xrightarrow{n\to\infty} f$ pointwise and $\sup_n ||f_n|| <
      \infty$. We write $f_n \xrightarrow{n\to\infty}_{bp} f$
  \end{definition}

  \begin{lemma}\label{lemma:bp}
    Let $(\Omega, \mathcal A, \mathbf P)$ be a probability (or finite)
    measure space, and $X, X_1, X_2,... : \Omega \to \mathbbm k$ such
    that $X_n \xrightarrow{n\to\infty}_{bp} X$. Then, $\mathbf E[X_n]
      \xrightarrow{n\to\infty} \mathbf E[X]$.
  \end{lemma}

  \begin{proof}
    \sloppy Note that the constant function $x \mapsto \sup_n ||f_n||$
    is integrable (since $\mathbf P$ is finite), so the result follows
    from dominated convergence.
  \end{proof}
  \switchcolumn
  \phantom{1}

  In lean this does not exist yet, although dominated
  convergence exists.\\ How can one formulate this kind of convergence
  using filters? \\ Task: ??

  \switchcolumn*[\section{Almost sure convergence and convergence in probability}]

  \begin{definition}
    Let $X,X_1,X_2,...$, all $E$-valued random variables.
    \begin{enumerate}
      \item We say that $X_n \xrightarrow{n\to\infty} X$ almost everywhere if
            $$\mathbf P(\lim_{n\to\infty} X_n = X) = 1.$$
            We also write $X_n\xrightarrow{n\to\infty}_{ae} X$.
      \item We say that $X_n \xrightarrow{n\to\infty} X$ in probability
            (or in measure) if, for all $\varepsilon>0$,
            $$ \lim_{n\to\infty} \mathbf P(r(X_n, X) > \varepsilon) = 0.$$
    \end{enumerate}
  \end{definition}

  \switchcolumn
  The two notions here are denoted \leanline{∀ᵐ (x : α) ∂P,
    Filter.Tendsto (fun n => X n x) Filter.atTop (nhds (X x))} and
  \leanline{MeasureTheory.TendstoInMeasure}, respectively.

  \switchcolumn*
  \begin{lemma}\label{l:aep}
    \sloppy Let $X,X_1,X_2,...$ be $E$-valued random variables with $X_n
      \xrightarrow{n\to\infty}_{ae} X$. Then, $X_n
      \xrightarrow{n\to\infty}_{p} X$.
  \end{lemma}

  \switchcolumn \sloppy This is called
  \leanline{MeasureTheory.tendstoInMeasure_of_tendsto_ae} in
  \leanline{mathlib}.

  \switchcolumn*
  \sloppy
  \begin{lemma}[\sloppy Uniqueness, limit in probability]\label{l:puni}
    \mbox{} Let $X,Y,X_1,X_2,...$ be $E$-valued random variables with
    $X_n \xrightarrow{n\to\infty}_{p} X$ and $X_n
      \xrightarrow{n\to\infty}_{p} Y$. Then, $X=Y$, almost surely.
  \end{lemma}

  \begin{proof}
    We write, using monotone convergence and Lemma~\ref{l:aep}
    \begin{align*}
      \mathbf P(X\neq Y) & = \lim_{\varepsilon \downarrow 0} \mathbf
      P(r(X,Y)>\varepsilon)                                          \\ & \leq \lim_{\varepsilon \downarrow 0}
      \lim_{n\to\infty}\mathbf P(r(X,X_n)>\varepsilon/2)             \\ & \quad +
      \mathbf P(r(Y,X_n)>\varepsilon/2) = 0.
    \end{align*}
  \end{proof}

  \switchcolumn

  This does not exist in mathlib yet.

  \switchcolumn*[\section{Separating algebras and characteristic functions}]

  \begin{definition}[Separating class of functions]
    \mbox{} Let $\mathcal M \subseteq \mathcal C_{b}(E,\mathbbm k)$.
    \begin{enumerate}
      \item If, for all $x,y\in E$ with $x\neq y$, there is
            $f\in\mathcal M$ with $f(x)\neq f(y)$, we say that $\mathcal M$
            separates points.
      \item
            If, for all $\mathbf P, \mathbf Q\in\mathcal P(E)$,
            $$ \mathbf P = \mathbf Q \quad \text{iff}\quad \mathbf P[f] =
              \mathbf Q[f] \text{ for all } f\in\mathcal M,$$ we say that
            $\mathcal M$ is separating in $\mathcal P(E)$.
      \item If (i) $1\in\mathcal M$ and (ii) if $\mathcal M$ is closed
            under sums and products, then we call $\mathcal M$ a
            (sub-)algebra.  If $\mathbbm k = \mathbb C$, and (iii) if
            $\mathcal M$ is closed under complex conjugation, we call
            $\mathcal M$ a star-(sub-)algebra.
    \end{enumerate}
  \end{definition}

  \switchcolumn
  In \leanline{mathlib}, 1.\ and 3.\ of the above definition are already
  implemented:

  \begin{minted}[highlightlines={}, mathescape, numbersep=5pt, framesep=5mm, bgcolor=mygray]{Lean}
  structure Subalgebra (R : Type u) (A : Type v) 
    [CommSemiring R] [Semiring A] [Algebra R A] 
    extends Subsemiring : Type v

  abbrev Subalgebra.SeparatesPoints {α : Type u_1} 
    [TopologicalSpace α] {R : Type u_2} 
    [CommSemiring R] {A : Type u_3} 
    [TopologicalSpace A] [Semiring A] [Algebra R A] 
    [TopologicalSemiring A] (s : Subalgebra R C(α, A))
    : Prop
\end{minted}

  The latter is an extension of \leanline{Set.SeparatesPoints}, which
  works on any set of functions.

  \switchcolumn*

  \noindent
  For the first result, we already need that $(E,r)$ has a metric
  structure.

  \begin{lemma}\label{l:unique}
    $\mathcal M := \mathcal C_b(E, \mathbbm k)$ is separating.
  \end{lemma}

  \begin{proof}
    We restrict ourselves to $\mathbbm k = \mathbb R$, since the result
    for $\mathbbm k = \mathbb C$ follows by only using functions with
    vanishing imaginary part. Let $\mathbf P, \mathbf Q \in \mathcal
      P(E)$. We will prove that $\mathbf P(A) = \mathbf Q(A)$ for all $A$
    closed. Since the set of closed sets is a $\pi$-system generating
    the Borel-$\sigma$-algebra, this suffices for $\mathbf P = \mathbf
      Q$. So, let $A$ be closed and $g = 1_A$ be the indicator
    function. Let $g_n(x) := (1 - n r(A,x))^+$ (where $r(A,y) := \inf_{y\in
        A}r(y,x)$) and note that $g_n(x) \xrightarrow{n\to\infty}
      1_A(x)$. Then, we have by dominated convergence
    \begin{align*}
      \mathbf P(A) & = \lim_{n\to\infty} \mathbf P[g_n] =
      \lim_{n\to\infty} \mathbf Q[g_n] = \mathbf Q(A),
    \end{align*}
    and we are done.
  \end{proof}
  \switchcolumn
  This is now implemented.

  \switchcolumn*
  We will use the Stone-Weierstrass Theorem below.
  Note that this requires $E$ to be compact.

  \switchcolumn

  \begin{minted}[highlightlines={0}, mathescape, numbersep=5pt, framesep=5mm, bgcolor=mygray, breaklines=true, breakafter=_]{Lean}
  theorem ContinuousMap.starSubalgebra_topologicalClosure_eq_top_of_separatesPoints
  {𝕜 : Type u_2} {X : Type u_1} [IsROrC 𝕜] [TopologicalSpace X] [CompactSpace X]
  (A : StarSubalgebra 𝕜 C(X, 𝕜)) (hA : Subalgebra.SeparatesPoints A.toSubalgebra) :
  StarSubalgebra.topologicalClosure A = ⊤
\end{minted}

  \switchcolumn*

  We also need (as proved in the last project) that compact sets are measurable.

  \switchcolumn

  \begin{minted}[highlightlines={0}, mathescape, numbersep=5pt, framesep=5mm, bgcolor=mygray, breaklines=true, breakafter=_]{Lean}
  theorem innerRegular_isCompact_isClosed_measurableSet_of_complete_countable
  [PseudoEMetricSpace α] [CompleteSpace α] [SecondCountableTopology α] [BorelSpace α]
  (P : Measure α) [IsFiniteMeasure P] :
  P.InnerRegular (fun s => IsCompact s ∧ IsClosed s) MeasurableSet
\end{minted}

  \switchcolumn*

  The proof of the following result follows
  \cite[Theorem~3.4.5]{EthierKurtz1986}.

  \begin{theorem}[Algebras separating points and separating algebras]
    \label{T:wc3}\mbox{}\\ \sloppy Let $(E,r)$ be a complete and separable extended pseudo-metric space, and
    $\mathcal M \subseteq\mathcal C_b(E,\mathbbm k)$ be a
    star-sub-algebra that separates points. Then, $\mathcal M$ is
    separating.
  \end{theorem}

  \begin{proof}
    Let $\mathbf P, \mathbf Q\in\mathcal P(E)$, $\varepsilon>0$ and $K$
    compact, such that $\mathbf P(K)>1-\varepsilon$, $\mathbf
      Q(K)>1-\varepsilon$, and $g \in \mathcal C_b(E,\mathbbm k)$. According to the
    Stone-Weierstrass Theorem, there is $(g_n)_{n=1,2,\dots}$ in
    $\mathcal M$ with
    \begin{align}
      \label{eq:wc9}
      \sup_{x\in K} |g_n(x) - g(x)| \xrightarrow{n\to\infty} 0.
    \end{align}
    So, (note that $C := \sup_{x\geq 0} xe^{-x^2} < \infty$)
    \begin{align*}
      \big|\mathbf P[ge^{-\varepsilon g^2}] - \mathbf Q[ge^{-\varepsilon
      g^2}] \big| & \leq \big|\mathbf P[ge^{-\varepsilon g^2}] -
      \mathbf P[ge^{-\varepsilon g^2};K] \big|                   \\ & \quad {} + \big|\mathbf
      P[ge^{-\varepsilon g^2};K] - \mathbf P[g_ne^{-\varepsilon
      g_n^2};K] \big|                                            \\ & \quad {} + \big| \mathbf P[g_ne^{-\varepsilon
      g_n^2};K] - \mathbf P[g_ne^{-\varepsilon g_n^2}] \big|     \\ &
      \quad {} + |\mathbf P[g_ne^{-\varepsilon g_n^2}] - \mathbf
      Q[g_ne^{-\varepsilon g_n^2}] \big|                         \\ & \quad {} + \big|\mathbf
      Q[g_ne^{-\varepsilon g_n^2}] - \mathbf Q[g_ne^{-\varepsilon
      g_n^2};K] \big|                                            \\ & \quad {}  + \big|\mathbf Q[g_ne^{-\varepsilon
      g_n^2}] - \mathbf Q[ge^{-\varepsilon g^2};K] \big|         \\ & \quad{} {}  +
      \big|\mathbf Q[ge^{-\varepsilon g^2};K] - \mathbf
      Q[ge^{-\varepsilon g^2}] \big|
    \end{align*}
    We bound the first term by
    $$\big|\mathbf P[ge^{-\varepsilon g^2}] - \mathbf P[ge^{-\varepsilon
              g^2};K] \big| \leq \frac{C}{\sqrt{\varepsilon}} \mathbf P(K^c)
      \leq C\sqrt{\varepsilon},$$ and analogously for the third, fifth and
    last. The second and second to last vanish for $n\to\infty$ due to
    \eqref{eq:wc9}. Since $\mathcal M$ is an algebra, we can
    approximate, using dominated convergence,
    \begin{align*}
      \mathbf P[g_ne^{-\varepsilon g_n^2}] & = \lim_{m \to \infty} \mathbf
      P[\underbrace{g_n \Big(1 - \frac{\varepsilon
      g_n^2}{m}\Big)^m}_{\in\mathcal M}\Big]                               \\ & = \lim_{m \to \infty}
      \mathbf Q[\underbrace{g_n \Big(1 - \frac{\varepsilon
              g_n^2}{m}\Big)^m}_{\in\mathcal M}\Big] = \mathbf
      Q[g_ne^{-\varepsilon g_n^2}],
    \end{align*}
    so the fourth term vanishes for $n\to\infty$ as well. Concluding,
    \begin{align*}
      \big|\mathbf P[g] - \mathbf Q[g] \big| = \lim_{\varepsilon\to 0}
      \big|\mathbf P[ge^{-\varepsilon g^2}] - \mathbf Q[ge^{-\varepsilon
              g^2}] \big| \leq 4C \lim_{\varepsilon \to 0}\sqrt{\varepsilon} =
      0.
    \end{align*}
    Since $g$ was arbitrary and $\mathcal C_b(E,\mathbbm k)$ is
    separating by Lemma~\ref{l:unique}, we find $\mathbf P = \mathbf Q$.
  \end{proof}
  \switchcolumn

  This does not exist in mathlib yet.

  \switchcolumn*

  We now come to characteristic functions and Laplace transforms.

  \begin{proposition}[Characteristic function unique]
    \label{Pr:char1}\mbox{}\\
    A probability measure $\mathbf P \in\mathcal P(\mathbb R^d)$ is
    uniquely given by its characteristic function.  \\ In other words,
    if $\mathbf P, \mathbf Q \in\mathcal P(\mathbb R^d)$ are such that
    $\int e^{itx} \mathbf P(dx) = \int e^{itx} \mathbf Q(dx)$ for all
    $t\in\mathbb R^d$. Then, $\mathbf P = \mathbf Q$.
  \end{proposition}

  \begin{proof}
    The set
    $$\mathcal M:= \Big\{ x\mapsto \sum_{k=1}^n a_k e^{i t_k x}; n \in
      \mathbb N, a_1,...,a_n \in \mathcal C, t_1,...,1_n\in\mathbb
      R^d\Big\}$$ separates points in $\mathbb R^d$. Since $\mathcal M
      \subseteq \mathcal C_b(\mathbb R^d,\mathbbm k)$ contains~1, is
    closed under sums and products, and closed under complex
    conjugation, it is a star-subalgebra of $\mathcal C_b(E,\mathbb
      C)$. So, the assertion directly follows from Theorem~\ref{T:wc3}.
  \end{proof}

  \switchcolumn This does not exist in mathlib yet.

\end{paracol}


\bibliographystyle{alpha}
\bibliography{main}


\end{document}
